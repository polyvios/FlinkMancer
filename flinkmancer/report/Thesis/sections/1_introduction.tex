\chapter{Introduction}


Traditionally, programs have been written for serial computation. That means a single problem is broken into smaller instructions and those instructions are executed sequentially, one at a time. Parallel computation is in the simplest way, the simultaneous execution of these instructions to compute the problem. It was introduced as way to model scientific problems, such as meteorology. This led to the design of parallel hardware and software which allowed the use of larger data sets. In this thesis we are trying to reproduce the feature extraction algorithm of TwitterMancer[1] using parallel computation. We aim to create a reliable way to extract features out of large scale data sets while also increasing the speedup on fixed size data sets.


\section{Challenges}


One of our main concerns was regarding the DRAM as our algorithm will produce a cross-product of N nodes, that will later be used to extract the features. Luckily, Flink is implemented in a way to make full use of the available memory without ever exceeding it and causing errors. 

The original goal was to use the extracted features and use them to train a Logistic regression model. However after version 1.8 Apache Flink no longer supports the Machine Learning library and is currently in early stage of implementation. Our solution was to extract the features, and create a file that could be used by another program to train the LR model. This led to another issue with our Filesystem memory as the output of our algorithm is a .csv document containing all features for each pair of nodes. Thus we have a file of ``N*N*Number of Features" (N = Unique Nodes) records taking a lot of space on our hard drives and eventually running out of space. We partially solved this for testing purposes using data sets that will output the maximum available size. In addition we created an alternative program, that produces a reduced output. While we cannot use this output, we are able to measure the time required to extract features out of even larger input data sets.
