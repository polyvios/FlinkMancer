\chapter{Run and Configurations}

In order to run the project, there are some requirements:
Download Apache Flink 1.10.1 for Scala 2.12

After the downloaded archive is unpacked, flink has to be configured for the standalone cluster. The ``/conf" directory has 2 important files.
\begin{itemize}
\item flink-conf.yaml:

In this file we set: 
\newline ``jobmanager.rpc.address"  to be the IP of Job Manager. 
\newline ``taskmanager.memory.flink.size" to be the highest available DRAM. 
\newline ``taskmanager.memory.off-heap.size" to at least 1024m. 
\newline ``taskmanager.numberOfTaskSlots" to the number of cores each machine has. 
\newline ``taskmanager.memory.network.min" and ``taskmanager.memory.network.max" 
\newline in a way that flink.size * network.fraction (0.1) is between min and max.

\item slaves:

In this file we add the id of each machine that will run a task manager. 
\end{itemize}
Now we can start our cluster and run the project
\begin{itemize}
\item  “./flink-1.10.0/bin/start-cluster.sh" 
\item  “./flink-1.10.0/bin/flink run flinkmancer.jar -{}-cores [number of total cores] -{}-path [path to dataset] -{}-outpath [path to output file inclunding the name]" 
\end{itemize}

The data sets needs to be in ./data/{layer} for each different layer used. \newline 
In our test runs we used 4 different layers, ``follow", ``reply", ``quote", ``retweet".
After the program is executed , it produces a Features.csv file, containing 100 Features about each Edge of Vectors (each Vector being a user) and their common neighbors. 
